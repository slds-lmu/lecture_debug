\documentclass[11pt,compress,t,notes=noshow, xcolor=table]{beamer}
\usepackage[]{graphicx}
% graphicx is loaded via lmu-lecture.sty as well
\usepackage[]{color}
% maxwidth is the original width if it is less than linewidth
% otherwise use linewidth (to make sure the graphics do not exceed the margin)
\makeatletter
\def\maxwidth{ %
  \ifdim\Gin@nat@width>\linewidth
    \linewidth
  \else
    \Gin@nat@width
  \fi
}
\makeatother

% ---------------------------------%
% latex-math dependencies, do not remove:
% - mathtools
% - bm
% - siunitx
% - dsfont
% - xspace
% ---------------------------------%

%--------------------------------------------------------%
%       Language, encoding, typography
%--------------------------------------------------------%

\usepackage[english]{babel}
\usepackage[utf8]{inputenc} % Enables inputting UTF-8 symbols
% Standard AMS suite (loaded via lmu-lecture.sty)
\usepackage{amsmath,amsfonts,amssymb}

% Font for double-stroke / blackboard letters for sets of numbers (N, R, ...)
% Distribution name is "doublestroke"
% According to https://mirror.physik.tu-berlin.de/pub/CTAN/fonts/doublestroke/dsdoc.pdf
% the "bbm" package does a similar thing and may be superfluous.
% Required for latex-math
\usepackage{dsfont}

% bbm – "Blackboard-style" cm fonts (https://www.ctan.org/pkg/bbm)
% Used to be in common.tex, loaded directly after this file
% Maybe superfluous given dsfont is loaded
% TODO: Check if really unused?
% \usepackage{bbm}

% bm – Access bold symbols in maths mode - https://ctan.org/pkg/bm
% Required for latex-math, preferred over \boldsymbol
% https://tex.stackexchange.com/questions/3238/bm-package-versus-boldsymbol
\usepackage{bm}

% pifont – Access to PostScript standard Symbol and Dingbats fonts
% Used for \newcommand{\xmark}{\ding{55}, which is never used
% aside from lecture_advml/attic/xx-automl/slides.Rnw
% \usepackage{pifont}

% Quotes (inline and display), provdes \enquote
% https://ctan.org/pkg/csquotes
\usepackage{csquotes}

% Adds arg to enumerate env, technically superseded by enumitem according
% to https://ctan.org/pkg/enumerate
% Replace with https://ctan.org/pkg/enumitem ?
% Even better: enumitem is not really compatible with beamer and breaks all sorts of things
% particularly the enumerate environment. The enumerate package also just isn't required
% from what I can tell so... don't re-add it I guess?
% \usepackage{enumerate}

% Line spacing - provides \singlespacing \doublespacing \onehalfspacing
% https://ctan.org/pkg/setspace
% \usepackage{setspace}

% mathtools – Mathematical tools to use with amsmath
% https://ctan.org/pkg/mathtools?lang=en
% latex-math dependency according to latex-math repo
\usepackage{mathtools}

% Maybe not great to use this https://tex.stackexchange.com/a/197/19093
% Use align instead -- TODO: Global search & replace to check, eqnarray is used a lot
% $ rg -f -u "\begin{eqnarray" -l | grep -v attic | awk -F '/' '{print $1}' | sort | uniq -c
%   13 lecture_advml
%   14 lecture_i2ml
%    2 lecture_iml
%   27 lecture_optimization
%   45 lecture_sl
\usepackage{eqnarray}

% For shaded regions / boxes
% Used sometimes in optim
% https://www.ctan.org/pkg/framed
\usepackage{framed}

%--------------------------------------------------------%
%       Cite button (version 2024-05)
%--------------------------------------------------------%

% Superseded by style/ref-buttons.sty, kept just in case these don't work out somehow.

% Note this requires biber to be in $PATH when running,
% telltale error in log would be e.g. Package biblatex Info: ... file 'authoryear.dbx' not found
% aside from obvious "biber: command not found" or similar.
% Tried moving this to lmu-lecture.sty but had issues I didn't quite understood,
% so it's here for now.

\usepackage{textcase} % for \NoCaseChange
\usepackage{hyperref}

% Only try adding a references file if it exists, otherwise
% this would compile error when references.bib is not found
% NOTE: Bibliography packages (usebib, biblatex) are now loaded by ref-buttons.sty when needed
% This keeps all bibliography-related setup in one place

% \newcommand{\citelink}[1]{%
% \NoCaseChange{\resizebox{!}{9pt}{\protect\beamergotobutton{\href{\usebibentry{\NoCaseChange{#1}}{url}}{\begin{NoHyper}\cite{#1}\end{NoHyper}}}}}%
% }

%--------------------------------------------------------%
%       Displaying code and algorithms
%--------------------------------------------------------%

% Reimplements verbatim environments: https://ctan.org/pkg/verbatim
% verbatim used sed at least once in
% supervised-classification/slides-classification-tasks.tex
% Removed since code should not be put on slides anyway
% \usepackage{verbatim}

% Both used together for algorithm typesetting, see also overleaf: https://www.overleaf.com/learn/latex/Algorithms
% algorithmic env is also used, but part of the bundle:
%   "algpseudocode is part of the algorithmicx bundle, it gives you an improved version of algorithmic besides providing some other features"
% According to https://tex.stackexchange.com/questions/229355/algorithm-algorithmic-algorithmicx-algorithm2e-algpseudocode-confused
\usepackage{algorithm}
\usepackage{algpseudocode}

%--------------------------------------------------------%
%       Tables
%--------------------------------------------------------%

% multi-row table cells: https://www.namsu.de/Extra/pakete/Multirow.html
% Provides \multirow
% Used e.g. in evaluation/slides-evaluation-measures-classification.tex
\usepackage{multirow}

% colortbl: https://ctan.org/pkg/colortbl
% "The package allows rows and columns to be coloured, and even individual cells." well.
% Provides \columncolor and \rowcolor
% \rowcolor is used multiple times, e.g. in knn/slides-knn.tex
\usepackage{colortbl}

% long/multi-page tables: https://texdoc.org/serve/longtable.pdf/0
% Not used in slides
% \usepackage{longtable}

% pretty table env: https://ctan.org/pkg/booktabs
% Is used
% Defines \toprule
\usepackage{booktabs}

%--------------------------------------------------------%
%       Figures: Creating, placing, verbing
%--------------------------------------------------------%

% wrapfig - Wrapping text around figures https://de.overleaf.com/learn/latex/Wrapping_text_around_figures
% Provides wrapfigure environment -used in lecture_optimization
\usepackage{wrapfig}

% Sub figures in figures and tables
% https://ctan.org/pkg/subfig -- supersedes subfigure package
% Provides \subfigure
% \subfigure not used in slides but slides-tuning-practical.pdf errors without this pkg, error due to \captionsetup undefined
\usepackage{subfig}

% Actually it's pronounced PGF https://en.wikibooks.org/wiki/LaTeX/PGF/TikZ
\usepackage{tikz}

% No idea what/why these settings are what they are but I assume they're there on purpose
\usetikzlibrary{shapes,arrows,automata,positioning,calc,chains,trees, shadows}
\tikzset{
  %Define standard arrow tip
  >=stealth',
  %Define style for boxes
  punkt/.style={
    rectangle,
    rounded corners,
    draw=black, very thick,
    text width=6.5em,
    minimum height=2em,
    text centered},
  % Define arrow style
  pil/.style={
    ->,
    thick,
    shorten <=2pt,
    shorten >=2pt,}
}

%--------------------------------------------------------%
%       Beamer setup and custom macros & environments
%--------------------------------------------------------%

% Main sty file for beamer setup (layout, style, lecture page numbering, etc.)
% For long-term maintenance, this may me refactored into a more modular set of .sty files
\usepackage{../../style/lmu-lecture}
% Custom itemize wrappers, itemizeS, itemizeL, etc
\usepackage{../../style/customitemize}
% Custom framei environment, uses custom itemize!
\usepackage{../../style/framei}
% Custom frame2 environment, allows specifying font size for all content
\usepackage{../../style/frame2}
% Column layout macros
\usepackage{../../style/splitV}
% \image and derivatives
\usepackage{../../style/image}
% New generation of reference button macros
\usepackage{../../style/ref-buttons}

% Used regularly
\let\code=\texttt

% Not sure what/why this does
\setkeys{Gin}{width=0.9\textwidth}

% -- knitr leftovers --
% Used often in conjunction with \definecolor{shadecolor}{rgb}{0.969, 0.969, 0.969}
% Removing definitions requires chaning _many many_ slides, which then need checking to see if output still ok
\definecolor{fgcolor}{rgb}{0.345, 0.345, 0.345}
\definecolor{shadecolor}{rgb}{0.969, 0.969, 0.969}
\newenvironment{knitrout}{}{} % an empty environment to be redefined in TeX

%-------------------------------------------------------------------------------------------------------%
%  Unused stuff that needs to go but is kept here currently juuuust in case it was important after all  %
%-------------------------------------------------------------------------------------------------------%

% \newcommand{\hlnum}[1]{\textcolor[rgb]{0.686,0.059,0.569}{#1}}%
% \newcommand{\hlstr}[1]{\textcolor[rgb]{0.192,0.494,0.8}{#1}}%
% \newcommand{\hlcom}[1]{\textcolor[rgb]{0.678,0.584,0.686}{\textit{#1}}}%
% \newcommand{\hlopt}[1]{\textcolor[rgb]{0,0,0}{#1}}%
% \newcommand{\hlstd}[1]{\textcolor[rgb]{0.345,0.345,0.345}{#1}}%
% \newcommand{\hlkwa}[1]{\textcolor[rgb]{0.161,0.373,0.58}{\textbf{#1}}}%
% \newcommand{\hlkwb}[1]{\textcolor[rgb]{0.69,0.353,0.396}{#1}}%
% \newcommand{\hlkwc}[1]{\textcolor[rgb]{0.333,0.667,0.333}{#1}}%
% \newcommand{\hlkwd}[1]{\textcolor[rgb]{0.737,0.353,0.396}{\textbf{#1}}}%
% \let\hlipl\hlkwb

% \makeatletter
% \newenvironment{kframe}{%
%  \def\at@end@of@kframe{}%
%  \ifinner\ifhmode%
%   \def\at@end@of@kframe{\end{minipage}}%
%   \begin{minipage}{\columnwidth}%
%  \fi\fi%
%  \def\FrameCommand##1{\hskip\@totalleftmargin \hskip-\fboxsep
%  \colorbox{shadecolor}{##1}\hskip-\fboxsep
%      % There is no \\@totalrightmargin, so:
%      \hskip-\linewidth \hskip-\@totalleftmargin \hskip\columnwidth}%
%  \MakeFramed {\advance\hsize-\width
%    \@totalleftmargin\z@ \linewidth\hsize
%    \@setminipage}}%
%  {\par\unskip\endMakeFramed%
%  \at@end@of@kframe}
% \makeatother

% \definecolor{shadecolor}{rgb}{.97, .97, .97}
% \definecolor{messagecolor}{rgb}{0, 0, 0}
% \definecolor{warningcolor}{rgb}{1, 0, 1}
% \definecolor{errorcolor}{rgb}{1, 0, 0}
% \newenvironment{knitrout}{}{} % an empty environment to be redefined in TeX

% \usepackage{alltt}
% \newcommand{\SweaveOpts}[1]{}  % do not interfere with LaTeX
% \newcommand{\SweaveInput}[1]{} % because they are not real TeX commands
% \newcommand{\Sexpr}[1]{}       % will only be parsed by R
% \newcommand{\xmark}{\ding{55}}%

% textpos – Place boxes at arbitrary positions on the LATEX page
% https://ctan.org/pkg/textpos
% Provides \begin{textblock}
% TODO: Check if really unused?
% \usepackage[absolute,overlay]{textpos}

% -----------------------%
% Likely knitr leftovers %
% -----------------------%

% psfrag – Replace strings in encapsulated PostScript figures
% https://www.overleaf.com/latex/examples/psfrag-example/tggxhgzwrzhn
% https://ftp.mpi-inf.mpg.de/pub/tex/mirror/ftp.dante.de/pub/tex/macros/latex/contrib/psfrag/pfgguide.pdf
% Can't tell if this is needed
% TODO: Check if really unused?
% \usepackage{psfrag}

% arydshln – Draw dash-lines in array/tabular
% https://www.ctan.org/pkg/arydshln
% !! "arydshln has to be loaded after array, longtable, colortab and/or colortbl"
% Provides \hdashline and \cdashline
% Not used in slides
% \usepackage{arydshln}

% tabularx – Tabulars with adjustable-width columns
% https://ctan.org/pkg/tabularx
% Provides \begin{tabularx}
% Not used in slides
% \usepackage{tabularx}

% placeins – Control float placement
% https://ctan.org/pkg/placeins
% Defines a \FloatBarrier command
% TODO: Check if really unused?
% \usepackage{placeins}

% Can't find a reason why common.tex is not just part of this file?
% This file is included in slides and exercises

% Rarely used fontstyle for R packages, used only in 
% - forests/slides-forests-benchmark.tex
% - exercises/single-exercises/methods_l_1.Rnw
% - slides/cart/attic/slides_extra_trees.Rnw
\newcommand{\pkg}[1]{{\fontseries{b}\selectfont #1}}

% Spacing helpers, used often (mostly in exercises for \dlz)
\newcommand{\lz}{\vspace{0.5cm}} % vertical space (used often in slides)
\newcommand{\dlz}{\vspace{1cm}}  % double vertical space (used often in exercises, never in slides)
\newcommand{\oneliner}[1] % Oneliner for important statements, used e.g. in iml, algods
{\begin{block}{}\begin{center}\begin{Large}#1\end{Large}\end{center}\end{block}}

% Don't know if this is used or needed, remove?
% textcolor that works in mathmode
% https://tex.stackexchange.com/a/261480
% Used e.g. in forests/slides-forests-bagging.tex
% [...] \textcolor{blue}{\tfrac{1}{M}\sum^M_{m} [...]
% \makeatletter
% \renewcommand*{\@textcolor}[3]{%
%   \protect\leavevmode
%   \begingroup
%     \color#1{#2}#3%
%   \endgroup
% }
% \makeatother



\title{SplitV Demo and Test Cases}

\begin{document}

\titlemeta{% Chunk title 
SplitV Demo
}{% Lecture title  
Advanced Layout Macros for Beamer
}{
figure_man/neo3_2.png
}{
\item Demonstrate splitV layout capabilities
\item Test integration with other environments
\item Showcase alignment and spacing features
}

% ------------------------------------------------------------------------------

\begin{frame}{SplitV Overview}
  \splitVTT{
  \textbf{Key Features}
  \begin{itemize}
    \item Works in all contexts including within lists
    \item Handles alignment consistently
    \item Improved spacing and layout
  \end{itemize}
  }{
  \textbf{Available Layouts}
  \begin{itemize}
    \item Two-column layouts with various alignments
    \item Three-column layouts with equal or custom widths
    \item Grid layouts for organizing content
  \end{itemize}
  }
  
  \vfill
  
  \splitVCC{
  \textbf{Implementation}
  \begin{itemize}
    \item Uses \code{lrbox} for maximum compatibility
    \item Simplified codebase for better reliability
    \item Optimized for beamer slides
  \end{itemize}
  }{
  \textbf{Benefits}
  \begin{itemize}
    \item Consistent visual appearance
    \item Easier to create complex layouts
    \item Works with all beamer features
  \end{itemize}
  }
\end{frame}

% ------------------------------------------------------------------------------

\begin{frame}{Standard Two-Column Layouts}
  \textbf{Center-Center Alignment (CC)}
  
  \splitVCC{
  \textbf{Left Column}
  \begin{itemize}
    \item This content is centered vertically
    \item Relative to the right column
    \item With equal distribution of space
  \end{itemize}
  }{
  \textbf{Right Column}
  
  This content is also centered vertically.
  
  \image[0.9]{figure_man/neo3_1.png}
  }
\end{frame}

% ------------------------------------------------------------------------------

\begin{frame}{Standard Two-Column Layouts 2}
  
  \textbf{Top-Top Alignment (TT)}
  
  \splitVTT[0.6]{
  \textbf{Left Column (60\% width)}
  \begin{itemize}
    \item This content aligns to the top
    \item Creating a different visual balance
    \item Width can be adjusted as needed
  \end{itemize}
  }{
  \textbf{Right Column (40\% width)}
  
  This content also aligns to the top.
  
  The vertical alignment is consistent regardless of content height.
  }
\end{frame}

% % ------------------------------------------------------------------------------

% \begin{frame}{Mixed Alignment Options}
%   \textbf{Top-Center Alignment (TC)}
  
%   \splitVTC{
%   \textbf{Left Column (Top)}
%   \begin{itemize}
%     \item This content aligns to the top
%     \item With multiple items
%     \item Creating a taller column
%   \end{itemize}
%   }{
%   \textbf{Right Column (Center)}
  
%   This content is vertically centered relative to the left column.
  
%   \image[0.9]{figure_man/neo3_3.png}
%   }
  
% \end{frame}

% % ------------------------------------------------------------------------------

\begin{frame2}[footnotesize]{Integration with Frame2 Environment}
  Using splitV inside the frame2 environment with small font.
  
  \splitVCC{
  \textbf{Left Column}
  \begin{itemize}
    \item The entire content has smaller font
    \item Due to the frame2 environment
    \item SplitV adapts correctly
  \end{itemize}
  }{
  \textbf{Right Column}
  
  \image[0.9]{figure_man/neo3_1.png}
  
  Image scaling works properly with the column width.
  }
  
  \vfill
  
  Notice that the spacing and alignment remain consistent despite the font size change.
\end{frame2}

% % ------------------------------------------------------------------------------

\begin{framei}[fs=footnotesize,sep=M]{Integration with framei Environment}
  \item The framei environment creates an itemize list
  \item We can nest splitV inside list items
  \item Using splitVCC within a list item:
  \splitVCC[0.7]{
  \item This continues the itemize list
  \item No need for a nested itemize
  \item The items appear properly beside the image
  }{
  \imageC[1][KITTEN]{figure_man/neo3_3.png}
  }
  
  \item Back to the main list
  \item With proper spacing between elements
\end{framei}

% % ------------------------------------------------------------------------------

\begin{frame}{SplitV Inside Regular Itemize}
  \begin{itemize}
    \item First regular bullet point
    \splitVCC[0.6]{
    \item This appears beside the image
    \item No need for a nested itemize environment
    }{
    \image[0.9]{figure_man/neo3_3.png}
    }
    
    \item We can continue the list afterward
    
    \item Another example with top alignment:
    \splitVTT[0.7]{
    \item Top-aligned content
    \item In a normal itemize environment
    }{
    \image[0.8]{figure_man/neo3_1.png}
    }
    
    \item Final regular bullet point
  \end{itemize}
\end{frame}

% % ------------------------------------------------------------------------------

\begin{frame}{Three-Column Layouts}
  \textbf{Equal Width Columns}
  
  \splitVThree{
  \textbf{Column 1}
  \begin{itemize}
    \item First item
    \item Second item
  \end{itemize}
  }{
  \textbf{Column 2}
  
  \image[0.95]{figure_man/neo3_1.png}
  }{
  \textbf{Column 3}
  \begin{itemize}
    \item Third column
    \item Last column
  \end{itemize}
  }
  
\end{frame}

% % ------------------------------------------------------------------------------

\begin{frame}{Three-Column Layouts 2}
  
  \textbf{Custom Width Columns}
  
  \splitVThreeC[0.25]{0.5}{0.25}{
  \textbf{25\%}
  
  Narrower column
  }{
  \textbf{50\%}
  
  \image[1]{figure_man/neo3_2.png}
  
  Wider middle column
  }{
  \textbf{25\%}
  
  Narrower column
  }
\end{frame}

% % ------------------------------------------------------------------------------

\begin{frame}{Grid Layouts}
  \textbf{Standard 2×2 Grid}
  
  \twobytwo{
  \image[0.9]{figure_man/neo3_1.png}
  }{
  \image[0.9]{figure_man/neo3_2.png}
  }{
  \image[0.9]{figure_man/neo3_3.png}
  }{
  \textbf{Text Cell}
  
  This cell contains text instead of an image
  }
\end{frame}

% % ------------------------------------------------------------------------------

\begin{frame}{Grid Layouts}
  
  \textbf{Custom Grid}
  
  \gridLayout[0.6]{
  \textbf{Wider Left Column}
  \begin{itemize}
    \item This column takes 60\% width
    \item Good for text-heavy content
  \end{itemize}
  }{
  \image[0.9]{figure_man/neo3_1.png}
  }{
  \image[0.9]{figure_man/neo3_1.png}
  }{
  \textbf{Bottom Right}
  
  Text in the bottom right cell
  }
\end{frame}

% % ------------------------------------------------------------------------------

\begin{frame}{Compact Layout}
  \textbf{splitVCompact Example}
  
  Standard layout for reference:
  \splitVCC[0.4]{
  \textbf{Left Column (40\%)}
  \begin{itemize}
    \item Regular splitVCC with 40\% width
    \item Uses full textwidth for layout
  \end{itemize}
  }{
  \textbf{Right Column (60\%)}
  \begin{itemize}
    \item Takes remaining width
    \item With proper spacing
  \end{itemize}
  }
  
\end{frame}

% % ------------------------------------------------------------------------------

% \begin{frame}{Compact Layout 2}
  
%   Compact layout with explicit widths:
%   \splitVCompact{0.38}{0.58}{
%   \textbf{Left Column (38\%)}
%   \begin{itemize}
%     \item Explicitly specified width
%     \item For both columns
%   \end{itemize}
%   }{
%   \textbf{Right Column (58\%)}
%   \begin{itemize}
%     \item Second width also specified
%     \item Columns don't need to use full width
%     \item Good for precise control
%   \end{itemize}
%   }
% \end{frame}

% ------------------------------------------------------------------------------

% \begin{frame}{Complex Nesting Example}
%   \begin{itemize}
%     \item This demonstrates a more complex nesting scenario

%     \item First level splitV:
%     \splitVCC[0.65]{
%       \item Left column with a nested splitV:
%       \splitVTT{
%         \item Inner left column
%         \item With multiple items
%       }{
%         \image[0.9]{figure_man/neo3_3.png}
%       }

%       \item Continuing the outer left column
%     }{
%       \textbf{Outer Right Column}

%       \image[0.9]{figure_man/neo3_1.png}
%     }

%     \item Back to the main list
%     \item Everything maintains proper spacing and alignment
%   \end{itemize}
% \end{frame}

\endlecture
\end{document}
