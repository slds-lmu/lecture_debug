\documentclass[11pt,compress,t,notes=noshow, xcolor=table]{beamer}
\input{../../style/preamble}
\input{../../latex-math/basic-math}


\title{Introduction to Debugging Lectures}

\begin{document}

\titlemeta{% Chunk title (example: CART, Forests, Boosting, ...), can be empty
  Demo Lecture
}{% Lecture title  
  Layout Macro Testing With Cannibalized Content
}{% Relative path to title page image: Can be empty but must not start with slides/
  figure/compboost-illustration-2.png
}{
  \item Test layout
  \item Fix stuff
  \item Go home
}
% ------------------------------------------------------------------------------

\begin{frame}{Demo Slides}
  This is a demo lecture chunk.

  It is recommended to view these slides as PDF and LaTeX code side by side

  Refer to the \code{slds-lmu/lecture\_debug} repository for the source and PDF:
  \begin{itemize}
    \item \href{https://slds-lmu.github.io/lecture_debug/lecture_debug/slides/demo/slides-demo-layout.pdf}{\beamergotobutton{Compiled PDF}}
    \item \href{https://github.com/slds-lmu/lecture_debug/blob/main/slides/demo/slides-demo-layout.tex}{\beamergotobutton{LaTeX source}}
  \end{itemize}

\end{frame}

\begin{frame}[allowframebreaks]{Whats inside the tin}

  Column layout helpers:

  \vfill

  \begin{itemize}
    \item \code{\textbackslash splitV} for two columns of equal / custom widths
    \item \code{\textbackslash splitVCompact}: Compact version of \code{\textbackslash splitV}
    \item \code{\textbackslash splitVThree}: 3-column layout (equal widths)
    \item \code{\textbackslash splitVThreeCustom}: Version of \code{\textbackslash splitVThree} with custom widths
    \item \code{\textbackslash twobytwo}: 2$\times$2 layout for quadrants, vertically centered
  \end{itemize}


  \framebreak

  Wrappers for \code{\textbackslash itemize} with different vertical spacing:

  \vfill

  \begin{itemize}
    \item \code{\textbackslash itemizeS}: \code{itemsep} is 1pt (less than default)
    \item \code{\textbackslash itemizeM}: \code{itemsep} is 5pt (slightly larger than default)
    \item \code{\textbackslash itemizeL}: \code{itemsep} is 15pt (larger than default)
    \item \code{\textbackslash itemizefill}: \code{itemsep} is \code{\\fill} to extend across the available vertical space
  \end{itemize}
\end{frame}

\begin{frame}[allowframebreaks]{Two columns, equal sizes}

  The \code{splitV} macro creates two equally wide columns:

  \vfill

  \code{\textbackslash splitV\{left content\}\{right content\}}

  \vfill

  \splitV{
    \textbf{Left column} Lorem ipsum dolor sit amet, consectetur adipiscing elit
  }{
    \textbf{Right column} Lorem ipsum dolor sit amet, consectetur adipiscing elit
  }

  \vfill

  Example with itemize environment left and text right:

  \splitV{
    \begin{itemize}
      \item Example itemize content
      \item Second itemize item
    \end{itemize}
  }{
    Lorem ipsum dolor sit amet, consectetur adipiscing elit
  }

  \vfill
  Note: The macro name \code{\textbackslash twocolumn} was already taken!

\end{frame}

\begin{frame}[allowframebreaks]{Two columns, unequal sizes}

  If you want to create columns with uneven widths, you only need to specify the width of the first column:

  \vfill

  \code{\textbackslash splitV[0.75]\{left content\}\{right content\}} creates a column with 75\% width and the
  second column will fill the remaining space:

  \vfill

  \splitV[0.75]{First column with 75\% of the text width on the slide}{Second column with rest}

  \vfill

  \splitV[0.2]{Tiny column}{Second column with a lot of room for activities}


\end{frame}
\begin{frame}[allowframebreaks]{Two columns, special cases}


  To enforce vertical centering of the content, use \code{\textbackslash splitVCentered[0.6]\{left content\}\{right content\}} analogously to \code{\textbackslash splitV}:

  \splitVCentered[0.6]{
    \begin{itemize}
      \item Example itemize content
      \item Second itemize item
      \item A third thing that is also important
      \item Do you like dogs?
      \item Hi mom
    \end{itemize}
  }{
    This text should be vertically centered relative to the left itemize list which is probably good
  }

  \framebreak

  For completeness, there is also a compact version if you do not want to use the full slide width:

  \code{\textbackslash splitVCompact\{0.2\}\{0.2\}} only takes up 20\% of the slide width in each column for a total of 40\% with a minimal horizontal spacer:

  \vfill

  \splitVCompact{0.20}{0.20}{First column with 20\%}{Second column with 20\%}


\end{frame}

\begin{frame}{3 columns}

  Using \code{\textbackslash splitVThree\{Left\}\{Middle\}\{Right\}} for 3 equally sized columns:

  \vfill

  \splitVThree{
    First column content is here
  }{
    Second column content is here as well
  }{
    And also a third column is here just in case
  }

  \vfill

  Then there is \code{\textbackslash splitVThreeCustom} for three columns of
  arbitrary widths, where the width of the third column can also be inferred from the first two
  \vfill

  \splitVThreeCustom[0.2]{0.2}{0}{
    first column with 20\% width
  }{
    second column with 20\% width
  }{
    third column with remaining width
  }

\end{frame}

% ------------------------------------------------------------------------------

\begin{frame}{2$\times$2 layout / quadrants}

  Use \code{\textbackslash twobytwo\{top left\}\{top  right\}\{bottom left\}\{bottom right\}} for horizontally aligned quadrants

  \vfill

  \twobytwo{%
    \begin{itemize}
      \item Example top left...
      \item ...quadrant content
      \item Next to a figure
    \end{itemize}
  }{%
    \includegraphics[width=\textwidth]{figure/boosting-cwb-blpool1.png}
  }{%
    \begin{itemize}
      \item Bottom left...
      \item ...quadrant content
    \end{itemize}
  }{%
    \includegraphics[width=\textwidth]{figure/boosting-cwb-blpool2.png}
  }

\end{frame}

\begin{frame}{2$\times$2 layout: All images}

  \vfill

  \twobytwo{%
    \includegraphics[width=\textwidth]{figure/boosting-cwb-blpool1.png}
  }{%
    \includegraphics[width=\textwidth]{figure/boosting-cwb-blpool2.png}
  }{%
    \includegraphics[width=\textwidth]{figure/boosting-cwb-blpool3.png}
  }{%
    \includegraphics[width=\textwidth]{figure/boosting-cwb-blpool4.png}
  }
\end{frame}

\begin{frame}{Disclaimer}


  \vfill

  Please note that all of these spacing and layout wrappers are unfortunately somewhat unstable and may lead to unexpected / undesired behavior within other environments or in combination with each other.

  \vfill

  For example, \code{\textbackslash splitV} uses negative spacing below it to prevent wasting space and overflowing to the next frame, but that means that you \textbf{will often need} to use \code{\textbackslash vfill} after \code{\textbackslash splitV} to get the desired vertical spacing.

  \vfill

  (In fairness, \code{\textbackslash vfill} should probably be used very liberally for vertical spacing anyway)

\end{frame}

% ------------------------------------------------------------------------------
% itemize wrapper
% ------------------------------------------------------------------------------

\begin{frame}[allowframebreaks]{itemize wrappers}

  Presets fro \code{itemize} with different vertical spacings (\code{itemsep}). The default value for \code{itemsep} is apparently \code{2pt}. Use \code{\textbackslash the \textbackslash itemsep} to find out the current value.


  \vfill

  \splitV{
    \begin{itemize}
      \item \textbf{Default itemize}
      \item itemsep is unmodified
      \item Another thing
      \item Words
      \item A fourth thing to show
      \item Just filling space here
      \item Hello there
    \end{itemize}
  }{
    \begin{itemizeS}
      \item \textbf{Uses environment} \code{itemizeS}
      \item itemsep is: \the\itemsep
      \item Another thing
      \item Words
      \item A fourth thing to show
      \item Just filling space here
      \item Hello there
    \end{itemizeS}
  }

  \framebreak

  \splitV{
    \begin{itemize}
      \item \textbf{Default itemize}
      \item itemsep is unmodified
      \item Another thing
      \item Words
      \item A fourth thing to show
      \item Just filling space here
      \item Hello there
    \end{itemize}
  }{
    \begin{itemizeM}
      \item \textbf{Uses environment} \code{itemizeM}
      \item itemsep is: \the\itemsep
      \item Another thing
      \item Words
      \item A fourth thing to show
      \item Just filling space here
      \item Hello there
    \end{itemizeM}
  }

  \framebreak

  \splitV{
    \begin{itemize}
      \item \textbf{Default itemize}
      \item itemsep is unmodified
      \item Another thing
      \item Words
      \item A fourth thing to show
      \item Just filling space here
      \item Hello there
    \end{itemize}
  }{
    \begin{itemizeL}
      \item \textbf{Uses environment} \code{itemizeL}
      \item itemsep is: \the\itemsep
      \item Another thing
      \item Words
      \item A fourth thing to show
      \item Just filling space here
      \item Hello there
    \end{itemizeL}
  }

  \framebreak


  \begin{itemizefill}
    \item \textbf{Uses environment} \code{itemizefill}
    \item itemsep is: \the\itemsep
    \item Automatically uses all vertical space
    \item \textbf{Warning:} This does not work in \code{\textbackslash splitV} (yet?)
  \end{itemizefill}

\end{frame}

\endlecture
\end{document}
