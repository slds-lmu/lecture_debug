\documentclass[11pt,compress,t,notes=noshow, xcolor=table]{beamer}
\input{../../style/preamble}
\input{../../latex-math/basic-math}

\title{Introduction to Debugging Lectures}

\begin{document}

\titlemeta{% Chunk title (example: CART, Forests, Boosting, ...), can be empty
Demo Lecture
}{% Lecture title  
Layout Macro Testing With Cannibalized Content
}{% Relative path to title page image: Can be empty but must not start with slides/
figure/compboost-illustration-2.png
}{
\item Test layout
\item Fix stuff
\item Go home
}
% ------------------------------------------------------------------------------

\begin{frame}{Demo Slides}
  This is a demo lecture chunk.
  
  It is recommended to view these slides as PDF and LaTeX code side by side
  
  Refer to the \code{slds-lmu/lecture\_debug} repository for the source and PDF:
  \begin{itemizeS}
    \item \href{https://slds-lmu.github.io/lecture_debug/lecture_debug/slides/demo/slides-demo-layout.pdf}{\beamergotobutton{Compiled PDF}}
    \item \href{https://github.com/slds-lmu/lecture_debug/blob/main/slides/demo/slides-demo-layout.tex}{\beamergotobutton{LaTeX source}}
  \end{itemizeS}
  
\end{frame}

\begin{frame}{\textbackslash splitV(TCB)(TCB)}
  
  \begin{itemize}
    \item \code{\textbackslash splitVCC} creates two centered columns
    \item \code{\textbackslash splitVTT} creates two top-aligned columns 
    \item \code{\textbackslash splitVBT} left is bottom-, right is top aligned
    
  \end{itemize}
  
  \vfill
  
  \splitVTT{
  \textbf{Left column} Lorem ipsum dolor sit amet, consectetur adipiscing elit
  }{
  \textbf{Right column} Lorem ipsum dolor sit amet, consectetur adipiscing elit
  }
  
  \vfill
  
  \splitVCC{
  \begin{itemize}
    \item Example itemize content
    \item Second itemize item
  \end{itemize}
  }{
  Lorem ipsum dolor sit amet, consectetur adipiscing elit
  }
  
\end{frame}

\begin{frame}{\textbackslash splitV with unequal sizes}
  
  You only need to specify the width of the first column:
  
  \vfill
  
  \code{\textbackslash splitVTT[0.75]\{left content\}\{right content\}} creates a column with 75\% width and the
  second column will fill the remaining space:
  
  \vfill
  
  \splitVTT[0.75]{First column with 75\% of the text width on the slide}{Second column with rest}
  
  \vfill
  
  \splitVTT[0.2]{Tiny column}{Second column with a lot of room for activities}
  
\end{frame}

% ------------------------------------------------------------------------------

\begin{frame}{Two columns, minimal adjustments}
  
  Compact version if you do not want to use the full slide width:
  
  \code{\textbackslash splitVCompact\{0.2\}\{0.2\}} only takes up 20\% of the slide width in each column for a total of 40\% with a minimal horizontal spacer:
  
  \vfill
  
  \splitVCompact{0.20}{0.20}{First column with 20\%}{Second column with 20\%}
  
  
\end{frame}

% ------------------------------------------------------------------------------

\begin{frame}{3 columns}
  
  Using \code{\textbackslash splitVThree\{Left\}\{Middle\}\{Right\}} for 3 equally sized columns:
  
  \vfill
  
  \splitVThree{
  First column content is here
  }{
  Second column content is here as well
  }{
  And also a third column is here just in case
  }
  
  \vfill
  
  Then there is \code{\textbackslash splitVThreeCustom} for three columns of
  arbitrary widths, where the width of the third column can also be inferred from the first two
  \vfill
  
  \splitVThreeCustom[0.2]{0.2}{0}{
  first column with 20\% width
  }{
  second column with 20\% width
  }{
  third column with remaining width
  }
  
\end{frame}

% ------------------------------------------------------------------------------

\begin{frame}{twobytwo 2$\times$2 layout / quadrants}
  
  Use \code{\textbackslash twobytwo\{top left\}\{top  right\}\{bottom left\}\{bottom right\}} for horizontally aligned quadrants
  
  \vfill
  
  \twobytwo{%
  \begin{itemize}
    \item Example top left...
    \item ...quadrant content
    \item Next to a figure
  \end{itemize}
  }{%
  \includegraphics[width=\textwidth]{figure/boosting-cwb-blpool1.png}
  }{%
  \begin{itemize}
    \item Bottom left...
    \item ...quadrant content
  \end{itemize}
  }{%
  \includegraphics[width=\textwidth]{figure/boosting-cwb-blpool2.png}
  }
  
\end{frame}

% ------------------------------------------------------------------------------

\begin{frame}{2$\times$2 layout: All images}
  
  \vfill
  
  \twobytwo{%
  \includegraphics[width=\textwidth]{figure/boosting-cwb-blpool1.png}
  }{%
  \includegraphics[width=\textwidth]{figure/boosting-cwb-blpool2.png}
  }{%
  \includegraphics[width=\textwidth]{figure/boosting-cwb-blpool3.png}
  }{%
  \includegraphics[width=\textwidth]{figure/boosting-cwb-blpool4.png}
  }
\end{frame}


% ------------------------------------------------------------------------------
% itemize wrapper
% ------------------------------------------------------------------------------

\begin{frame}[allowframebreaks]{itemize wrappers}
  
  Presets fro \code{itemize} with different vertical spacings (\code{itemsep}). The default value for \code{itemsep} is apparently \code{2pt}. Use \code{\textbackslash the \textbackslash itemsep} to find out the current value.
  
  
  \vfill
  
  \splitVTT{
  \begin{itemize}
    \item \textbf{Default itemize}
    \item itemsep is unmodified
    \item Another thing
    \item Words
    \item A fourth thing to show
    \item Just filling space here
    \item Hello there
  \end{itemize}
  }{
  \begin{itemizeS}
    \item \textbf{Uses environment} \code{itemizeS}
    \item itemsep is: \the\itemsep
    \item Another thing
    \item Words
    \item A fourth thing to show
    \item Just filling space here
    \item Hello there
  \end{itemizeS}
  }
  
  \framebreak
  
  \splitVTT{
  \begin{itemize}
    \item \textbf{Default itemize}
    \item itemsep is unmodified
    \item Another thing
    \item Words
    \item A fourth thing to show
    \item Just filling space here
    \item Hello there
  \end{itemize}
  }{
  \begin{itemizeM}
    \item \textbf{Uses environment} \code{itemizeM}
    \item itemsep is: \the\itemsep
    \item Another thing
    \item Words
    \item A fourth thing to show
    \item Just filling space here
    \item Hello there
  \end{itemizeM}
  }
  
  \framebreak
  
  \splitVTT{
  \begin{itemize}
    \item \textbf{Default itemize}
    \item itemsep is unmodified
    \item Another thing
    \item Words
    \item A fourth thing to show
    \item Just filling space here
    \item Hello there
  \end{itemize}
  }{
  \begin{itemizeL}
    \item \textbf{Uses environment} \code{itemizeL}
    \item itemsep is: \the\itemsep
    \item Another thing
    \item Words
    \item A fourth thing to show
    \item Just filling space here
    \item Hello there
  \end{itemizeL}
  }
  
  \framebreak
  
  
  \begin{itemizeF}
    \item \textbf{Uses environment} \code{itemizeF}
    \item itemsep is: \the\itemsep
    \item Automatically uses all vertical space
  \end{itemizeF}
  
\end{frame}

% ------------------------------------------------------------------------------

\begin{frame}{Image taking the full slide width}
  \image{figure/gbm_sine.png}
\end{frame}

% ------------------------------------------------------------------------------

\begin{frame}{Full width image with attribution}
  \imageC[1][CITEKEY]{figure/gbm_sine.png}
\end{frame}

% ------------------------------------------------------------------------------

\begin{frame}{Images with alignment and optional attribution}
  
  \imageL[0.4][CITEKEY]{figure/gbm_sine.png}
  \imageC[0.4][CITEKEY]{figure/gbm_sine.png}
  \imageR[0.4][CITEKEY]{figure/gbm_sine.png}
  
  
\end{frame}

% ------------------------------------------------------------------------------

\begin{frame}{Images within itemize}
  
  \begin{itemize}
    \item Foo bar
    \item Plubber
    \imageC[0.5]{figure/boosting-cwb-blpool1.png}
    \item jfowiehfgnsdlkjnfg
  \end{itemize}
  
  
\end{frame}

% ------------------------------------------------------------------------------

\begin{frame}[allowframebreaks]{Images within splitV}
  
  \splitVTT{
  \image{figure/boosting-cwb-blpool1.png}
  }{
  \image{figure/boosting-cwb-blpool1.png}
  }
  
  \vfill
  
  \splitVCC[0.4]{
  \image{figure/boosting-cwb-blpool1.png}
  }{
  \image{figure/boosting-cwb-blpool1.png}
  }
  
\end{frame}

% ------------------------------------------------------------------------------

\begin{frame}[allowframebreaks]{Images within splitVCC}
  
  \splitVCC{
  \image[0.5]{figure_man/neo3_2.jpg}
  }{
  \image[0.7]{figure_man/neo3_2.jpg}
  }
  
  \vfill
  
  \splitVCC[0.4]{
  \imageC[1][KITTEN]{figure_man/neo3_2.jpg}
  }{
  \image{figure_man/neo3_2.jpg}
  }
  
\end{frame}

% ------------------------------------------------------------------------------

\begin{framei}{frame environment}
  \item This is a frame consisting only of an itemize environment
  \item Nothing else here, just itemize.
  \item No extra options yet for alignment or sizing
\end{framei}

\endlecture
\end{document}
