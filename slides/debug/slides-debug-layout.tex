\documentclass[11pt,compress,t,notes=noshow, xcolor=table]{beamer}
\input{../../style/preamble}
\input{../../latex-math/basic-math}
\input{../../latex-math/basic-ml}
\input{../../latex-math/ml-ensembles.tex}
\input{../../latex-math/ml-trees.tex}

\usepackage{transparent}

\title{Introduction to Debugging Lectures}

\begin{document}

\titlemeta{% Chunk title (example: CART, Forests, Boosting, ...), can be empty
    Debug Lecture
  }{% Lecture title  
    Debug Layout Chunk With Cannibalized Content
  }{% Relative path to title page image: Can be empty but must not start with slides/
  figure/compboost-illustration-2.png
  }{
  \item Test layout
  \item Fix stuff
  \item Go home
}
% ------------------------------------------------------------------------------

\begin{vbframe}{Vertical splits with splitV (2 columns)}

This uses the \code{splitV} macro to create to equally wide columns:

\code{\textbackslash splitV\{left\}\{right\}} for 2 equally sized columns:

\vfill

\splitV{
  \textbf{Left column} lorem ipsum dolor sit amet
}{
  \textbf{Right column} lorem ipsum dolor sit amet
}

\vfill

\code{\textbackslash splitV[0.75]} for example to create a column with 75\% width and the
 second column to fill the remaining space:

\vfill

\splitV[0.75]{First column with 75\% of the text width on the slide}{Second column with rest}

\vfill

And lastly a compact version \code{\textbackslash splitVCompact\{0.2\}\{0.2\}} to only take up the width specificed and minimal margins all around:

\vfill

\splitVCompact{0.20}{0.20}{First column with 20\%}{Second column with 20\%}

\vfill


\end{vbframe}

\begin{vbframe}{Vertical splits  (3 columns)}

Using \code{\textbackslash splitVThree} for 3 equally sized columns and \code{\textbackslash splitVThreeCustom} for three columns of 
arbitrary widths, where the width of thee third column can also be inferred from the first two.
\vfill

\splitVThree{
  first column
}{
  second column
}{
  third column
}

\vfill

\code{\textbackslash splitVThreeCustom} must be specified with the widths of the columns as arguments, but the last 
one can be set to 0 to infer the remaining width.

\vfill

\splitVThreeCustom[0.2]{0.2}{0}{
  first column with 20\% width
}{
  second column with 20\% width
}{
  third column with remaining width
}



\end{vbframe}

% ------------------------------------------------------------------------------

\begin{vbframe}{2x2 layout}

Uses \code{\textbackslash twobytwo\{left\}\{right\}\{top\}\{bottom\}} for hrizontally aligned quadrants

\vfill

\twobytwo{%
 \begin{itemize}
    \item A1
    \item B1
    \item C1
    \item D1
 \end{itemize}
}{%
\includegraphics[width=\textwidth]{figure/boosting-cwb-blpool1.png}
}{%
\begin{itemize}
  \item A2
  \item B2
\end{itemize}
}{%
\includegraphics[width=\textwidth]{figure/boosting-cwb-blpool2.png}
}

\end{vbframe}

\begin{vbframe}{2x2 layout: All images}

  \vfill
  
  \twobytwo{%
  \includegraphics[width=\textwidth]{figure/boosting-cwb-blpool1.png}
  }{%
  \includegraphics[width=\textwidth]{figure/boosting-cwb-blpool2.png}
  }{%
  \includegraphics[width=\textwidth]{figure/boosting-cwb-blpool3.png}
  }{%
  \includegraphics[width=\textwidth]{figure/boosting-cwb-blpool4.png}
  }
  \end{vbframe}

\endlecture
\end{document}
